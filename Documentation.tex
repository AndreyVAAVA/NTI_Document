\documentclass[a4paper, 12pt]{article}
\usepackage{cmap}
\usepackage[T2A]{fontenc}
\usepackage[utf8]{inputenc}
\usepackage[english,russian]{babel}
\usepackage{amsmath,amsfonts,amsthm,mathtools}
\usepackage{icomma}
\usepackage{euscript}
\usepackage{mathrsfs}
\usepackage{hyperref}
\usepackage{xcolor}
\definecolor{linkcolor}{HTML}{534B4F}
\definecolor{urlcolor}{HTML}{6699CC}
\definecolor{citecolor}{HTML}{e68a00}
\hypersetup{pdfstartview=FitH,  linkcolor=linkcolor,urlcolor=urlcolor,citecolor=citecolor, colorlinks=true}
\author{Андрей Волков, София Красова, Алиев Марат}
\title{Документация и описание процесса}
\date{\today}

\begin{document}
	\maketitle
	\part*{Описание методов и классов}
	\section{Классы}
	\subsection{CourseGroup}
	Используется для форматирования JSON в класс.
	\subsection{CourseList}
	Используется для форматирования JSON в класс.
	\subsection{Course}
	Используется для форматирования JSON в класс.
	\subsection{Main}
	
	
	\part*{Описание процессов}
	
	1
	
	В переменную url присваивается ссылка на API.
	
	В connection мы формируем запрос к серверу.
	
	В переменную in мы получем в читателя(точнее в виде класса читателя(BufferedReader), который помогает считать информацию из потока connection.getInputStream()).
	
	После мы создаём StringBuffer, который в разы проще помогает состовлять строки из других строк.
	
	Потом мы считываем данные(цикл while ((inputLine = in.readLine()) != null)) и сразу же после считывания закрываем поток.
	
	2
	
	Создаём переменную типа Gson, которая поможет спарсить JSON. В следующей строке парсим.
	
	Создаём список url, который называем requestsById.
	
	Создаём список курсов(courses), который называем courses.
	
	Из-за lambda создаёи финальную(да без final, но мы её не меняем потом) переменную, finalUrl.
	
	Потом при помощи lambda выражения и foreach заполняем requestsById ссылками(так как мы должны сформатировать, мы это делаем через цикл). Помимо этого мы отсеиваем уроки, которые не были выложены.
	
	3
	
	Проходим циклом по всем запросам
	
	Внутри цикла повторяем процесс 1-2 и добавляем в курс список курсов(courses).
	
	Повторяем процесс 2-3, но вместо создания requestsById, мы его чистим и слегка меняем данные при заполнении в foreach(можно понять почему, посмотрев на сайте документацию). И вместо списка курсов, мы создаём список в который записваем данные о группе курса.
	
	Проходим циклом по всем запросам
	
	Внутри цикла повторяем процесс 1-2.
	
	После получения массива(благодаря Gson), мы переводим его в список и через foreach+lamda для заполнения данных в Excel мы заполняем список из массивов типа Object.
	
	Используем классы из библиотеки Apache.POI.
	
	А именно:
	\begin{enumerate}
		\item Создаём таблицу.
		\item Проходя по ранее созданому списку из массивов мы создаём foreach цикл.
		\item Проходим по массиву из списка и записываем данные в Excel. (перед этим апкастим Object к нужному типу)
	\end{enumerate}
\end{document}