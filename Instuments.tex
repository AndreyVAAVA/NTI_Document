\documentclass[a4paper, 12pt]{article}
\usepackage{cmap}
\usepackage[T2A]{fontenc}
\usepackage[utf8]{inputenc}
\usepackage[english,russian]{babel}
\usepackage{amsmath,amsfonts,amsthm,mathtools}
\usepackage{icomma}
\usepackage{euscript}
\usepackage{mathrsfs}
\usepackage{hyperref}
\usepackage{xcolor}
\definecolor{linkcolor}{HTML}{534B4F}
\definecolor{urlcolor}{HTML}{6699CC}
\definecolor{citecolor}{HTML}{e68a00}
\hypersetup{pdfstartview=FitH,  linkcolor=linkcolor,urlcolor=urlcolor,citecolor=citecolor, colorlinks=true}
\author{Андрей Волков, София Красова, Алиев Марат}
\title{Инструменты используемые на спецпроекте НТИ\cite{NTI}}
\date{\today}

\begin{document}
	\maketitle
	\part{Используемые инструменты}
	\section{Java\cite{Java}}
	\subsection{Преимущества:}
		\begin{enumerate}
			\item Статическая типизация
			\item Большое количество библиотек
			\item JVM\cite{JVM}(другие языки, к примеру работают на JVM\cite{JVM}(Kotlin\cite{Kotlin} самый яркий пример))
		\end{enumerate}

	\section{SQL\cite{SQL} или NoSQL\cite{NoSQL} или Excel\cite{Excel}}
	\subsection{Преимущества:}
		\begin{enumerate}
			\item Удобный формат хранения
			\item Легко представлять информацию
			\item Наличие высокоуровневых библиотек и разных ORM\cite{ORM}(к примеру Android Room\cite{Room})
		\end{enumerate}
	\subsection{SQL\cite{SQL}}
		\subsubsection{Преимущества:}
			\begin{enumerate}
				\item Это язык программирования
				\item Проще всего хранить огромные массивы данных
			\end{enumerate}
		\subsubsection{Минусы:}
		\begin{enumerate}
			\item Нужно знать язык запросов
		\end{enumerate}
	
	\subsubsection{NoSQL\cite{NoSQL}}
		\subsubsection{Преимущества:}
			\begin{enumerate}
				\item Формат хранения JSON\cite{JSON}(из-за чего его легко читать)
			\end{enumerate}
		\subsubsection{Минусы:}
		\begin{enumerate}
			\item Мало мест где используют как основной инструмент
			\item Чуть большая сложность в сравнении с SQL\cite{SQL} при хранении огромных массивов данных
		\end{enumerate}
	
	\subsection{Excel\cite{Excel}}
		\subsubsection{Преимущества:}
			\begin{enumerate}
				\item Популярность(практически каждый знает как пользоваться Excel\cite{Excel})
				\item Самый красивый GUI\cite{GUI} интерфейс для работы с данными
			\end{enumerate}
		\subsubsection{Минусы:}
			\begin{enumerate}
				\item Сложность хранения больших массивов данных. (неудобство обработки)
				\item Проприетарные форматы
			\end{enumerate}
	
	\section{Infogram\cite{Infogram} или JavaFX\cite{JavaFX}(там же JFreeChart\cite{JFreeChart}), или GraphicsView\cite{GraphicsView}(и подобные под Android)}
	\subsection{Преимущества:}
		\begin{enumerate}
			\item Удобство в использовании.
			\item Красивое графическое представление.
		\end{enumerate}
	\subsection{JavaFX\cite{JavaFX}(и JFreeChart\cite{JFreeChart}) и Android библиотеки}
		\subsubsection{Преимущества:}
		\begin{enumerate}
			\item Использование кода для генерации графиков на основе данных.
			\item Использование меньшего количества инструментов.
		\end{enumerate}
		\subsubsection{Минусы:}
			\begin{enumerate}
				\item Необходимость знать Java\cite{Java}.
				\item Лишние зависимости в программу.
			\end{enumerate}
	
	\begin{thebibliography}{9}
		\bibitem{NTI}
		\href{https://nti-contest.ru/tracks/olimpiada-dlya-shkolnikov-8-11-klassov/proekt-novoy-sredy-zhizni/tekhnologicheskoe-predprinimatelstvo-spetsproekt/}{НТИ}
		\bibitem{Java}
		\href{https://www.java.com/ru/}{Java}
		\bibitem{JVM}
		\href{https://ru.wikipedia.org/wiki/Java_Virtual_Machine}{JVM}
		\bibitem{Kotlin}
		\href{https://kotlinlang.org/}{Kotlin}
		\bibitem{SQL}
		\href{https://ru.wikipedia.org/wiki/SQL}{SQL}
		\bibitem{NoSQL}
		\href{https://ru.wikipedia.org/wiki/NoSQL}{NoSQL}
		\bibitem{Excel}
		\href{https://www.microsoft.com/ru-ru/microsoft-365/excel}{Excel}
		\bibitem{ORM}
		\href{https://ru.bmstu.wiki/ORM_(Object-Relational_Mapping)}{ORM}
		\bibitem{Room}
		\href{https://developer.android.com/jetpack/androidx/releases/room}{Android Room}
		\bibitem{JSON}
		\href{https://ru.wikipedia.org/wiki/JSON}{JSON}
		\bibitem{GUI}
		\href{https://ru.wikipedia.org/wiki/Graphical_user_interface}{GUI}
		\bibitem{Infogram}
		\href{https://infogram.com/}{Infogram}
		\bibitem{JavaFX}
	
		\href{https://openjfx.io/}{JavaFX}
		\bibitem{JFreeChart}
		\href{https://www.jfree.org/jfreechart/api/javadoc/}{JFreeChart}
		\bibitem{GraphicsView}
		\href{http://developer.alexanderklimov.ru/android/simplepaint.php}{GraphicsView}
	\end{thebibliography}
\end{document}