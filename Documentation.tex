\documentclass[a4paper, 12pt]{article}
\usepackage{cmap}
\usepackage[T2A]{fontenc}
\usepackage[utf8]{inputenc}
\usepackage[english,russian]{babel}
\usepackage{amsmath,amsfonts,amsthm,mathtools}
\usepackage{icomma}
\usepackage{euscript}
\usepackage{mathrsfs}
\usepackage{hyperref}
\usepackage{xcolor}
\definecolor{linkcolor}{HTML}{534B4F}
\definecolor{urlcolor}{HTML}{6699CC}
\definecolor{citecolor}{HTML}{e68a00}
\hypersetup{pdfstartview=FitH,  linkcolor=linkcolor,urlcolor=urlcolor,citecolor=citecolor, colorlinks=true}
\author{Андрей Волков, София Красова, Алиев Марат}
\title{Документация и описание процесса}
\date{6 апреля 2021 г.}

\begin{document}
	\maketitle
	\part*{Описание методов и классов}
	\section{Классы}
	\subsection{CourseGroup}
	Используется для форматирования JSON в класс.
	\subsection{CourseList}
	Используется для форматирования JSON в класс.
	\subsection{Course}
	Используется для форматирования JSON в класс.
	\subsection{Main}
	Метод connectionStream - принимает url и после обрабатывает url, создаёт json и делает запрос на сервер webinar, после чего получает данные в поток и на следующих строках в цикле их из него, и уже после назначает все данные в переменную resoponse.
	
	\part*{Описание процессов в Main}
	Создаём переменную url, которая содержит ссылку, после вызываем connectionStream(url). Создаётся объект, состоящий из нескольких курсов.()
	
	Создаём список для запросов(requestsById) и список курсов(finalUrl нужен только для того, чтобы без проблем использовать лямбду). Заполняем requestsById запросами, которые будем делать, при этим исключая некоторые(если не было курса, то запрос в список не добавляется). Через цикл несколько раз вызываем connectionStream передавая каждый элемент из requestsById. После чего добавляем в список курсов данные полученные после каждого запроса.
	
	Чистим requestsById. Заполняем requestsById запросами. Создаём список с данными о группе курса(courseGroup) и список состоящий из массиво(objects)в типа Object(от него все наследуются). Через цикл несколько раз вызываем connectionStream передавая каждый элемент из requestsById. После чего добавляем в courseGroup данные полученные после каждого запроса. После заполняем objects, для того, чтобы добавить эти данные в Excel.(есть некоторые проверки, чтобы не дать полям быть пустыми(ну и есть один форматтер, из-за того, что иначе Excel не захочет работать))
	
	Добавляем данные в Excel.(вероятнее всего это будет позже перенесено в тело цикла)
	
	*Все sout сделаны для логирования и да, я знаю, что есть logd(Log.d()), но с sout проще. И про jUnit тесты знаю, но их написать пока не успел.(скоро будут)
\end{document}